 \documentclass[12pt]{article}
\title{FTP2P}
\author{Tony Riemer}
\date{\today}
\begin{document}
\maketitle
\section{Abstract}
A decentralized social media protocol

\section{Introduction}
\textbf{Data ownership}: In the current paradigm accept for internet services, data ownership is granted to the service itself.
Particularly, "free" services who store, analyze, and sell your data. The user is the product. 
With FtP2P, the publisher can own the data, determine where it is stored, and who can access it.
\linebreak
\linebreak
\textbf{Moderation + Monetization}: Moderation/censorship of content is ultimately at the discretion of the owner of the platform.
The may not reflect the desires of the users of the platform.
With a blockchain based platform, consensus can be reached in terms of rules/conduct, within certain sets of conditions.
e.g. youtubers who are demonetized for playing 'eye of the tiger' on piano while advertisers are free to display lewd/sexually explicit materials to minors.
\linebreak
\linebreak
\textbf{Accountability}
By nature of a blockchain, the record of every transaction recorded in a block is \textit{immutable}. That said, a node can never 'take back' an action (e.g. remove a post, a comment, an upload, a message to another node, etc), and a node can never deny having taken an action (authored a transaction).

\section{State}
The state is the representation of all transactions up to any particular hash applied on top of the genesis block.
\subsection{Genesis Block}
\subsection{Transaction}
\subsection{Block}
\subsection{Manifest}
\subsection{Node}

\section{API}
TODO

\section{Sync}
TODO

\end{document}