\RequirePackage{snapshot}
\documentclass [10pt, fancyhdr, twoside] {article}
\usepackage{float, graphicx, caption, amssymb, natbib}
\usepackage[usenames,dvipsnames]{color}
\usepackage{tabulary}
\usepackage [left=2.5cm, top=2.5cm, bottom=2.5cm, right=3cm] {geometry}  %% see geometry.pdf on how to lay out the page. There's lots.
\geometry{a4paper} %% or letter or a5paper or ... etc
\usepackage{fancyhdr}
\usepackage{xcolor}
\usepackage[scaled]{helvet}
\renewcommand*\familydefault{\sfdefault} %% Only if the base font of the document is to be sans serif

\usepackage[left]{lineno}
\usepackage[yyyymmdd,hhmmss]{datetime}

\renewcommand{\linenumberfont}{\normalfont\tiny\color{gray}}


\pagestyle{fancy}

\fancyhead{}
\fancyfoot{}

\fancyhead[RO,LE]{Heading}
\fancyfoot[RO,LE]{Page-\thepage}
\fancyfoot[C]{\textbf{Footer}}
\fancyfoot[RE, LO]{Created: \today\ at \currenttime}

\usepackage{blindtext}

\newcounter {note}
\stepcounter{note}

\renewcommand{\abstractname}{Abstract Name}

\newcommand {\Note} [1] {
    \marginpar {
        \tiny {
            {\color{gray}{\thenote  \  #1}}
            }
        }
    \stepcounter {note}
}

\newcommand {\MNote} [1] {
    \marginpar {
        \tiny {
            {\color{gray}{#1 }}
            }
        }
}

\begin{document}

\title{Mercury}
\author {Tony Riemer}

\date{\today}

\maketitle

\begin{abstract}
The intention is to allow nodes to define state transition machines that exist within the blockchain (think of this like ethereum's smart contracts, in a way). Nodes can then interact with the state transition machine in order to mutate the state. The idea is to have node-defined slices of state that can be mutated by other nodes. This state itself should be a blockchain. Transaction verification/validation should happen in real time, as well block creation in a fair manner.
\end{abstract}

\linenumbers

\section{Section}
\subsection{Sub Section}
\blindtext
\Note{First Note}

\subsection{Sub Section}
\blindtext

\subsection{Sub Section}
\blindtext

\subsection{Sub Section}
\Note{Second Note}
\blindtext

\section{Section}
\subsection{Sub Section}
\blindtext
\subsection{Sub Section}
\blindtext
\subsection{Sub Section}
\blindtext

\section{Section}
\subsection{Sub Section}
\blindtext
\subsection{Sub Section}
\blindtext
\subsection{Sub Section}
\blindtext
\Note{Third Note}


\end{document}